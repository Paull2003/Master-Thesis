% !TeX root = ../Praxisphasenbericht 1.tex

\chapter{Einleitung} \label{chpt:Einleitung}

% Unterkapitel
Mit der steigenden Leistungsdichte bei elektrischen Antrieben, wie zum Beispiel in elektrisch angetriebenen Fahrzeugen, ist es notwendig leistungsstarke Sicherheitseinrichtungen zu entwerfen. Dabei ist ein wichtiger Bestandteil das Erproben und Verifizieren dieser. Dabei sind die Anforderungen an die Messeinrichtungen groß, da in Testfällen große Ströme und Spannungen auftreten. Zudem ist eine gute Zeitauflösung notwendig, um die schnelle Änderung in diesen Größen festzustellen. 

Ziel dieser Arbeit ist die Konzeption der Steuerung an einem Stoßstromprüfstand. Zunächst wird daher der Entwurf der Steuerung des Prüfstands zum Test von Schutzeinheiten. Dies geschieht auf Basis einer bereits vorhandenen Steuerung, die im Zuge der Erweiterung auch modifiziert werden soll. Zur Integration des wird daher das Steuerungs- und Sicherheitskonzept betrachtet und überarbeitet. Dabei werden verschiedene kleinere Schaltungen entworfen, um die verschiedenen Funktionen des Stoßstromprüfstands zu steuern und sicheres Arbeiten zu gewährleisten. Anschließend erfolgt die Konzeption der Messwerterfassung, wobei Ströme und Spannungen erfasst werden sollen. Dabei müssen zum einen die Messeinrichtungen Konzeptioniert werden, zum anderen müssen erfasste Messdaten ausgewertet werden.  

Die Arbeit leistet damit einen Beitrag zur Prüfung und Erprobung von Sicherheitseinrichtungen in einem definierten und reproduzierbaren Umfeld. Darüber hinaus werden Konzepte zur Messung von Strömen und Spannungen erprobt.


%Die zunehmende Leistungsdichte in elektrischen Antrieben, wie sie beispielsweise in elektrisch angetriebenen Fahrzeugen zu beobachten ist, bedingt die Entwicklung leistungsstarker Sicherheitseinrichtungen. Ein signifikanter Aspekt in diesem Zusammenhang ist die Erprobung und Verifizierung der genannten Komponenten. Die Anforderungen an die Messeinrichtungen sind dabei als hoch einzustufen, da im Rahmen der Testfälle hohe Ströme und Spannungen auftreten. Darüber hinaus ist eine hohe Zeitauflösung erforderlich, um die schnellen Veränderungen in diesen Größenordnungen zu erfassen. 

%Die Zielsetzung dieser Arbeit besteht in der Konzeption einer Steuerung für einen Stoßstromprüfstand. In der vorliegenden Arbeit wird zunächst der Entwurf der Steuerung des Prüfstands zum Test von Schutzeinheiten dargelegt. Dieser Prozess erfolgt auf Grundlage einer bereits implementierten Steuerung, die im Rahmen der Erweiterung einer Modifikation unterzogen werden soll. In Anbetracht dessen erfolgt eine Überarbeitung des Steuerungs- und Sicherheitskonzepts, um dessen Integration zu gewährleisten. Zu diesem Zweck werden verschiedene kleinere Schaltungen entworfen, um die verschiedenen Funktionen des Stoßstromprüfstands zu steuern und ein sicheres Arbeiten zu gewährleisten. Im Anschluss erfolgt die Konzeption der Messwerterfassung, wobei die Erfassung von Strömen und Spannungen vorgesehen ist. Im Rahmen des Forschungsprojekts ist es erforderlich, dass die Messeinrichtungen konzipiert werden. Zudem ist eine Auswertung der erfassten Messdaten notwendig.  

%Die vorliegende Arbeit leistet einen Beitrag zur Prüfung und Erprobung von Sicherheitseinrichtungen in einem definierten und reproduzierbaren Umfeld. Darüber hinaus werden Konzepte zur Messung von Strömen und Spannungen erprobt.