% Deutsche Anpassungen %%%%%%%%%%%%%%%%%%%%%%%%%%%%%%%%%%%%%%%%%%%%%%%%%%%%%
\usepackage[T1]{fontenc}%     %% Schriftkodierung
%\usepackage[ansinew]{inputenc}%% Eingabekodierung
\usepackage[utf8]{inputenc}%% Eingabekodierung

\usepackage[ngerman]{babel}%  %% Neue deutsche Rechtschreibung verwenden
\addto\captionsngerman{%      %% Name des Literaturverzeichnisses verändern
  \renewcommand{\bibname}{Quellen}%
}
\usepackage{csquotes}
\usepackage{booktabs}
\usepackage{color}  % Farben im allgemeinen
\usepackage{colortbl}
\definecolor{rot}{rgb}{1,0.3,0.3}
\definecolor{drot}{rgb}{1,0.4,0.2}
\definecolor{grun}{rgb}{0.3,1,0.3}

\usepackage{pifont}
\usepackage{siunitx}
\sisetup{
	locale = DE ,
	per-mode = symbol
}

% URL %%%%%%%%%%%%%%%%%%%%%%%%%%%%%%%%%%%%%%%%%%%%%%%%%%%%%%%%%%%%%%%%%%%%%%
\usepackage{url}%               %% Links durch \url{}

% Schriftarten %%%%%%%%%%%%%%%%%%%%%%%%%%%%%%%%%%%%%%%%%%%%%%%%%%%%%%%%%%%%%
\usepackage{lmodern}% 			%% Latin-modern Schriftart

% Diverses %%%%%%%%%%%%%%%%%%%%%%%%%%%%%%%%%%%%%%%%%%%%%%%%%%%%%%%%%%%%%%%%%
%\usepackage{blindtext}

%% Textformatierung %%%%%%%%%%%%%%%%%%%%%%%%%%%%%%%%%%%%%%%%%%%%%%%%%%%%%%%%%
\usepackage[                  %% Package für Farbe
   pdftex,                    % Ausgabe für pdfTeX
   svgnames,                  % Zusätzliche Farben per Name hinzufügen
   x11names                   % noch mehr Farben
]{xcolor}%                 

%% Packages für Grafiken & Abbildungen %%%%%%%%%%%%%%%%%%%%%%%%%%%%%%%%%%%%%%
\usepackage[pdftex]{graphicx}% %% Zum Laden von Grafiken
\graphicspath{{Bilder/}}%     % Standardpfad für Bilder/Grafiken

\usepackage{fancybox}         % Ermöglicht zeichnen von Boxen von definierter Länge

% Bild statt Abbildung
\addto\captionsngerman{\renewcommand{\figurename}{Abbildung}} % with ngerman

\usepackage{float}
% fancy cross-referencing \fref & \Fref
\usepackage[german, plain]{fancyref}

\usepackage[
   format = hang,             % Ausrichtung
   labelformat = default,     % Bezeichnerformatierung
   labelsep = colon,          % Bezeichner-Trennzeichen
   justification = justified, % Text als Blocksatz setzen
   singlelinecheck = true,    % 1-Zeilen-Beschriftungen --> zentrieren erlaubt
   font = {it, small},        % Beschriftungsformat   
   labelfont = bf,            % Bezeichnerformat (Bild x.y...)
   margin = 0pt,              % Beschriftungsrand (auch links/rechts möglich {0pt, 0pt})
   twoside,                   % Zweiseitiger Textsatz
   parskip = 5pt,             % Abstand zwischen Absätzen in Beschriftungen
   skip = 8pt,                % Abstand von Beschriftung
   listfigurename = Bilderverzeichnis,  % Abbildungen anstatt Abbildungsverzeichnis
   listtablename  = Tabellen,           % Tabellen anstatt Tabellenverzeichnis
   figurewithin = chapter,    % Zählerbegrenzung festlegen (also Bild 3.1 anstatt Bild 43)
   tablewithin = chapter      % Zählerbegrenzung festlegen
]{caption}

\usepackage{afterpage}
\usepackage{subcaption}       %% Verwendung von subcaption statt subfigure, da subfigure veraltet ist
                              
%% Tabellen %%%%%%%%%%%%%%%%%%%%%%%%%%%%%%%%%%%%%%%%%%%%%%%%%%%%%%%%%%%%%%%%%%%%%%%%%
\usepackage{array}            %% Standarderweiterung für Tabellen   


%% Bibliographiestil %%%%%%%%%%%%%%%%%%%%%%%%%%%%%%%%%%%%%%%%%%%%%%%%%%

\usepackage[backend=biber,style=numeric,sorting=none]{biblatex}
\addbibresource{Kapitel/Literatur/Literatur.bib}



%%Packages für Kopf- und Fußzeile%%%%%%%%%%%%%%%%%%%%%%%%%%%%%%%%%%%%%%
\usepackage{fancyhdr}%

%%Packages für Fußnoten%%%%%%%%%%%%%%%%%%%%%%%%%%%%%%%%%%%%%%%%%%%%%%
\usepackage{chngcntr}%
\counterwithout*{footnote}{chapter} % Deaktiviert das Zurücksetzen des Fußnoten-Counters

%% Mathematik-Packages%%%%%%%%%%%%%%%%%%%%%%%%%%%%%%%%%%%%%%%%%%%%%%%%%
\usepackage{amsmath}             %% AMS Mathematikumgebungen/-erweiterungen   
\usepackage{amssymb}             %% AMS Symbol/Font-Paket
\usepackage{accents}             %% Erweiterung für Akzente im Mathematik-Modus
\usepackage{mathcomp}            %% Symbolerweiterungen für Mathematikmodus
\usepackage{amsthm}
\usepackage{amsbsy}

%% Erscheinungsbild von Paragraph ändern %%%%%%%%%%%%%%%%%%%%%%%%%%%%%%
% titlesec wurde entfernt, da es mit KOMA-Script in Konflikt steht
\RedeclareSectionCommand[beforeskip=6pt, afterskip=6pt]{section}
\RedeclareSectionCommand[beforeskip=5pt, afterskip=5pt]{subsection}
\RedeclareSectionCommand[beforeskip=3pt]{paragraph}

%% Abkürzungsverzeichnis %%%%%%%%%%%%%%%%%%%%%%%%%%%%%%%%%%%%%%%%%%%%%%%  
\usepackage[]{acronym}

\usepackage{upgreek}
\usepackage{setspace} 

%%Hyperlink-Package%%%%%%%%%%%%%%%%%%%%%%%%%%%%%%%%%%%%%%%%%%%%%%%%%%%%%
%% ACHTUNG: Als letztes package einfügen!
\usepackage[
   plainpages=false,           %% Arabische Zeichen für Link-Darstellung
   bookmarksopen=true,        %% Lesezeichenbaum aufklappen
   bookmarksopenlevel=1,      %% Aufklapptiefe
   pdfborder={0 0 0},         %% Rahmenfarbe
   pdfsubject={Thema des Dokuments}, %% PDF Thema
   pdfauthor={Dein Name},     %% Author
   pdfpagemode=UseOutlines,	  %% Ansicht im Adobe Reader
   pdftex                     %% Ausgabe mit pdfTeX
]{hyperref}

% Anpassen der Seitengeometrie, um die typearea Warnung zu beheben
\KOMAoptions{DIV=12}
\usepackage{multirow}
\usepackage{makecell}
